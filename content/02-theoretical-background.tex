% Local IspellDict: en
\chapter{Theoretical Background}\label{ch02:theoretical}

\section{Image Segmentation}

Image segmentation is the process of dividing an image into different regions by grouping pixels and assigning each pixel a label. 
This step is an important part of many computer vision applications, such as detecting tumors in medical images or identifying 
pedestrians in autonomous driving. According to human visual perception, the identified regions are non-overlapping and meaningful 
- however, defining what exactly counts as a “meaningful” region can be difficult, as human perception is subjective and 
object boundaries are not always clear \citep{yu_techniques_2023}.

There are three common types of segmentation:

\textit{Semantic segmentation} assigns every pixel in an image a semantic label, such as “car” or “sky”.
\textit{Instance segmentation} separates individual objects within the same class, for example distinguishing several people in one image.
\textit{Panoptic segmentation} combines both approaches by providing pixel-wise class labels and also identifying individual object instances.

\begin{figure}[h]
    \centering
    \includegraphics[width=0.45\textwidth]{figures/segmentation_types.png}
    \caption{Types of image segmentation by \cite{kirillov_panoptic_2019}}
\end{figure}

Earlier approaches to image segmentation include algorithms such as k-means-clustering \citep{dhanachandra_image_2015}. Yet in recent years, deep learning models 
have significantly improved the segmentation effect and performance, therefore becoming the dominant method for solving segmentation 
tasks in complex environments \citep{minaee_image_2022}.

\section{Saliency Segmentation}

\section{Models}

%%% Local Variables:
%%% mode: latex
%%% TeX-master: "../main_thesis"
%%% TeX-command-extra-options: "-shell-escape"
%%% End:
