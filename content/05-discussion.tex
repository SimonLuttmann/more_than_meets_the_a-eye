\section{Discussion}

\subsection{Limitations}

During our research we identified several limitations that shape how the results should be interpreted. These limitations reflect both technical challenges and aspects related 
to the nature of social media content. First, social media images show a high level of complexity. They include a wide range of content, contexts and formats. This diversity 
makes it difficult to define what should be considered salient across different situations. It also reduces the generalisability of models that are trained on more uniform 
data sources.

Second, we encountered technical constraints. Modern transformer based models that support tasks such as image captioning are often very large and require considerable 
computational resources. In many practical settings these models cannot be used directly or must be simplified, which can reduce performance. Third, the ground truth data 
available in existing datasets reflects subjective human judgement. What is perceived as salient can differ between individuals, which introduces uncertainty into training 
and evaluation. Finally, there are currently no datasets designed specifically for salient object detection in the context of social media. Most available datasets are 
collected in more controlled environments and do not capture the characteristics of social media content. This limits the ability to develop and evaluate models that address 
this specific setting.

\subsection{Future Research}

Future research can build on the findings of this seminar and explore areas that lie beyond its scope. One important direction is the development of datasets that focus 
specifically on social media content. Such datasets would give models the opportunity to learn from examples that reflect the diversity, style and context that are 
characteristic of images shared online. Another promising direction is the investigation of lightweight models for social media image analysis. Since many existing approaches 
rely on large and resource intensive architectures, identifying smaller and more efficient models could improve accessibility and practical use.

Further work may also examine how multimodal data can support salient object detection. Textual elements such as image captions can provide additional context and may enhance 
model performance when integrated with visual information. Moreover, researchers could explore new applications of salient object detection within social media analysis. 
This may include areas such as content understanding, user behaviour studies and automated moderation, where identifying salient elements may enable deeper insights and more 
effective tools.