% Local IspellDict: en
\chapter{Methodology}\label{ch03:method}

\section{Experimental Design}



\section{Planned Practical Steps}

The planned next steps can be divided into two main parts. The first part focuses on the optimization of the saliency object detection process, while the second part 
focuses on the extraction of insights from the segmented objects, based on an exploratory data analysis \citep{tukey_exploratory_1977}.

As discussed in the previous section, the current approach offers several options for improvement. These include image preprocessing, the adjustment of threshold values, 
the use of box prompts instead of point prompts, and the exploration of additional clustering methods such as k means. Moreover, it is possible to add further steps between 
the main stages of the workflow in order to create more stable and consistent results. As a last possibility, the model can be fine-tuned, using existing scientific datasets such as
WXSOD or PASCAL-S \citep{quan_wxsod_2025, ccvl_pascal-s_2018}, which already include saliency masks. The overall goal of this first step is to produce the best possible saliency masks for social 
media images. Better masks allow for a more reliable extraction of insights in the next step.

After the optimization step, the quality of the segmented masks must be evaluated. This evaluation includes three main questions. First, it must be examined if the masks truly 
represent the salient parts of the image. Second, the mask quality must be assessed by measuring how well the masks fit the expected important regions. Third, it must be 
analyzed whether certain types of images work better or worse with saliency based segmentation. This helps identify strengths and limitations of the approach for different 
kinds of social media content.

Furthermore, the generated masks allow for additional exploratory fields of analysis that can be grouped into three areas: user insights, content insights, and accessibility support. The user 
related area includes predictive user profiling, where salient regions help identify which visual elements attract individual users. The content related area includes a deeper 
understanding of social media images, the detection of clickbait content, and guidance for creators who work with under optimized images. The accessibility related area focuses 
on focus aware alternative text generation, where the masks help identify the most important visual elements for users with visual impairments. Together, these groups show the 
wider potential of the approach beyond the core segmentation step.

%%% Local Variables:
%%% mode: latex
%%% TeX-master: "../main_thesis"
%%% TeX-command-extra-options: "-shell-escape"
%%% End:
